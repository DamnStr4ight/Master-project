\chapter{Introduction}
    

    
     
     
     \section{Background}
        Ever since the oil crisis of 2014, which further amplified the shock of the financial crash of 2008, we have seen a shift in the energy industry, as well as in the geopolitical scene, from fossil fuels towards renewable energy. Big oil companies such as Statoil ASA are now investing heavily into the renewable energy market in order to assure their financial position in the future.\cite{StatoilRenewable}
        
        On a global scale, renewable energy is a yet largely untapped resource waiting to be exploited. A key advantage is its scale-ability. One can get everything from small mW solar panels to  large  MW rated wind turbines and GW rated hydro plants.
        
        Even though wind power already account for around 7\% of the worlds total power production, roughly 432 GW, \cite{WorldEnergyR}, it has a far greater potential. Some papers have the feasible peak production some where around 14 TW \cite{lu2009global}, while others have a more pessimistic calculations of around 1 TW \cite{de2011global}. The World energy council predicts that by 2030, the total installed wind energy capacity worldwide will reach 977 GW \cite{WorldEnergyR}.
        
        A challenge with wind power versus many other energy resources is that one has to generate it where the wind is, and this is rarely places which are densely populated. It has been reported that some countries, such as Denmark has in periods covered all its electricity needs with wind energy \cite{GreenTech}. It is furthermore host to one of the major wind turbine producers in the world, Vestas. In 2017 they tested a 9MW turbine, which broke the 24H world record for most power produced by a single wind turbine \cite{GreenTech}.
        
     \todo{read and include this : https://www.nrk.no/finnmark/statnett-vil-bygge-gasskraftverk-der-vindmoller-er-verdens-mest-effektive-1.13366177}
     \todo{Read and include this: https://www.nrk.no/finnmark/japan-vil-ha-energi-fra-finnmark-1.12378100}
     \todo{Read and include this: https://www.tu.no/artikler/industri-japanerne-skriker-etter-ren-energi-na-vil-de-hente-den-fra-finnmark/222173}
     \todo{Read and include this: https://www.nrk.no/sapmi/stor-ettersporsel-etter-vindkraft-1.7579288}
    
    


    \section{Formål og problemstilling}
    Optimal control of a wind‐farm hydrogen buffer system
    We have detailed data for wind production in a 45 MW wind park, which the owners want to extend
    to 200 MW. Trouble is, they cannot export more than 95 MW to the main grid because they are far
    from strong grid connections – and no one is going to strengthen them. The area does not have any
    hydro power, nor any other obvious way to store energy.
    One day, an engineer proposes the idea that all this power could be used to produce hydrogen, that
    they might use forzero‐emission transport in their new hydrogen cars, or re‐electrify and sell as power
    when there is no wind, or ship it to customers overseas.
    This thesis will include the following tasks:
    \begin{itemize}
        \item Generation of wind power production profiles, based on real data;
        \item Dynamic modelling of the system components:
        \begin{itemize}
            \item Wind turbines
            \item Electrolysers (efficient alkaline and fast PEM technologies)
            \item Fuel cells
            \item Storage units (hydrogen tanks)
        \end{itemize}
        \begin{itemize}
            \item Core task: define a high‐level control algorithm (MPC) to maximise profit, able to function with
    multiple constraints and to give in to the right one:
        \begin{itemize}
            \item Maintain energy independence when there is no wind;
            \item Meet hydrogen production quotas by certain dates;
            \item Maintain hydrogen availability for cars in face of uncertain refuelling schedule.
        \end{itemize}
        \item This thesis will make ample use of optimization techniques coupled with multiple uncertainties (wind,
    customers).
    
        \end{itemize}
        
    \end{itemize}
     
     
    
    
    
    
    
    \section{Rapportens oppbygging}
    
    
    
    
    
    \section{Omfang og avgrensninger}



